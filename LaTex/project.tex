\documentclass{article}
\usepackage[greek,english]{babel}
\usepackage[utf8]{inputenc}
\usepackage{alphabeta}
\usepackage{graphicx}

\begin{document}
% \sffamily
% \ttfamily

\title{E - Broker: Μια λεπτομερής ανάλυση}
\author{
	Ιωάννης Χείλαρης - 1115201500176
	\\
	Ιωάννης Μαλιάρας - 1115201500084
}

\maketitle

\newpage
\tableofcontents


\newpage
\section{Εισαγωγή}
	Στα πλαίσια αυτής της εργασίας, θα σχεδιάσουμε μια εφαρμογή Ιστού για τη διαβίβαση εντολών αγοράς και πώλησης μετοχών μιας εταιρίας.
	Αρχικά, θα μελετήσουμε τις απαιτήσεις του παραπάνω συστήματος με τη χρήση μεθόδων δομημένης ανάλυσης.

	
\newpage
\section{Δομημένη Ανάλυση}
	\subsection{Γενικό Διάγραμμα Ροής Δεδομένων}

		\subsection*{Παραδοχές}
		\begin{itemize}
			\item Το ΧΑΑ προμηθεύει το σύστημα E-Broker με όλα τα δεδομένα των μετοχών που διατίθενται προς αγοραπωλησία.
			\item Το ΧΑΑ ενημερώνει το σύστημα E-Broker περί της διαθεσιμότητας του (σε περίπτωση αργίας), αλλά και όλων των μετοχών που διατίθεται προς αγοραπωλησία
		\end{itemize}

		Τα 2 παραπάνω συνοψίζονται στο βέλος "Δεδομένα Μετοχών".

		\begin{figure}[!h]
			\includegraphics[width=\linewidth]{../Structured_Analysis/General_Diagram.png}
		\end{figure}

	\subsection{Διάγραμμα Ροής Δεδομένων Επιπέδου 1}

\newpage
\section{Επίλογος}



\end{document}